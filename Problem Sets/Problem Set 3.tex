%%%%%%%%%%%%%%%%%%%%%%%%%%%%%%%%%%%%%%%%%%%%%%%%%%%%%%%%%%%%%%%
%
% Welcome to Overleaf --- just edit your LaTeX on the left,
% and we'll compile it for you on the right. If you open the
% 'Share' menu, you can invite other users to edit at the same
% time. See www.overleaf.com/learn for more info. Enjoy!
%
%%%%%%%%%%%%%%%%%%%%%%%%%%%%%%%%%%%%%%%%%%%%%%%%%%%%%%%%%%%%%%%
\documentclass[11pt, article, oneside]{memoir}

% Some useful packages
\usepackage
{
  microtype,
  blindtext,
  multirow,
  makecell,
  graphicx,
  hyperref,
  nicematrix,
  mathtools,
  amsthm,
  amssymb,
  thmtools,
  bm,
  siunitx,
  caption
}

%% PDF Metadata
\hypersetup
{
  hidelinks,
  pdftitle    = Title,
  pdfauthor   = Ashnah Khalid Khan
}


% Page Layout for Printing A4
\settrimmedsize{297mm}{210mm}{*}
\setlength{\trimtop}{0pt}
\setlength{\trimedge}{\stockwidth}
\addtolength{\trimedge}{-\paperwidth}
\settypeblocksize{634pt}{448.13pt}{*}
\setulmargins{4cm}{*}{*}
\setlrmargins{*}{*}{1}
\setmarginnotes{17pt}{51pt}{\onelineskip}
\setheadfoot{\onelineskip}{2\onelineskip}
\setheaderspaces{*}{2\onelineskip}{*}
\checkandfixthelayout%


% Page Style
\makepagestyle{SD}
\nouppercaseheads
\makeoddhead{SD}{\textsc{Theory of Automata}}{}{\textsc{Summer 2023}}
\makeoddfoot{SD}{}{\thepage}{}
\makeheadrule{SD}{\textwidth}{\normalrulethickness}
% \makefootrule{SD}{\textwidth}{\normalrulethickness}
\pagestyle{SD}

% Paragraph Indent
\nonzeroparskip
\setlength{\parindent}{0pt}

% Command to make writing sets easier
\newcommand{\set}[1]{\{\, #1\, \}}

% Absolute Value Operator
\DeclarePairedDelimiter\abs{\lvert}{\rvert}

\begin{document}
% -------------
\begin{center}
\LARGE{\textsc{Problem Set 3}}
 
\large{\textsc{Ms. Asma Sanam Larik}}
\end{center}

   % 1. Design of Turing Machine single tape. 
   % 2. Design of Multitape TM
   % 3. Proof on equivalence of various models of TM
   % 4. Some decidable languages and their proofs
   % 5. Examples of Undecidable languages
   % 6. Design of enumerator as Turing recognizable language
   % 7. One example of Turing Unrecognizable language
   % 8. Reducibility of one problem to another
   % 9. Computation of  Time Complexity of some algorithms
   % 10. Some examples of NP Complete problems.

\begin{enumerate}
    % 1. Design of Turing Machine single tape. 
    \item
    \textbf{\{Design of Turing Machine single tape\}}
    \\Design a Turing machine that accepts the language \(L = \set{ bb\Sigma^* \mid \Sigma = \set{a, b}}\). Assume that the transition function \( \delta \) for the TM is defined as \( \delta : Q \times \Gamma \rightarrow Q \times \Gamma \times \set{L,R}\).

    % 2. Design of Multitape TM
    \item
    \textbf{\{Design of Multitape TM\}}

    % 3. Proof on equivalence of various models of TM
    \item
    \textbf{\{Proof on equivalence of various models of TM\}}
    \\Consider a Turing machine that, on any particular move, can either change the tape symbol or move the read-write head, but not both.
    \\(a) Give a formal definition of such a machine.
    \\(b) Show that the class of such machines is equivalent to the class of standard Turing machines.

    % 4. Some decidable languages and their proofs
    \item
    \textbf{\{Some decidable languages and their proofs\}}
    \\(a) Let \(A = \set{\langle R, S\rangle \mid \text{R and S are regular expressions and }L(R) \subseteq L(S)}\). Show that A is decidable.
    \\(b) Consider the problem of determining whether a DFA and a regular expression are equivalent. Express this problem as a language B and show that B is decidable.

    % 5. Examples of Undecidable languages
    \item
    \textbf{\{Examples of Undecidable languages\}}
    \\Let \(T = \set{\langle M\rangle \mid \text{M is a TM that accepts }w^R \text{ whenever it accepts w}}\). Show that T is undecidable.

    % 6. Design of enumerator as Turing recognizable language
    % 8. Reducibility of one problem to another
    \item
    \textbf{\{Design of enumerator as Turing recognizable language, Examples of Undecidable languages, Reducibility of one problem to another\}}
    \\Let \(E^{TM} = \set{\langle M\rangle \mid \text{M is a TM and }L(M) = \emptyset}\). Show that \(\overline{E^{TM}}\), the complement of \(E^{TM}\), is:
    \\(a) Turing-recognizable, using an enumerator.
    \\(b) undecidable, using a reduction from another undecidable language.

    % 7. One example of Turing Unrecognizable language
    % 8. Reducibility of one problem to another
    \item
    \textbf{\{One example of Turing Unrecognizable language, Examples of Undecidable languages, Reducibility of one problem to another\}}
    \\Let \(I = \set{\langle M\rangle \mid \text{M is a TM and L(M) is an infinite language}}\). Show that \(I\) is:
    \\(a) undecidable, using a reduction from another undecidable language.
    \\(b) Turing-\textbf{un}recognizable, using a reduction from another \textbf{un}recognizable language.

    % 9. Computation of  Time Complexity of some algorithms
    % 10. Some examples of NP Complete problems.
    \item
    \textbf{\{Computation of  Time Complexity of some algorithms, Some examples of NP Complete problems\}}
    \\Assess the time complexity of the following problems to determine if they are in the class P, NP and/or NP-Complete. Prove your answer to each part.
    \\(a) \(C = \set{\langle G\rangle \mid \text{G is a connected undirected graph}}\).
    \\(b) \(D = \set{\langle G, k\rangle \mid \text{G has a dominating set with }k\text{ nodes}}\)
    \\NOTE: A subset of the nodes of a graph G is a dominating set if every other node of G is adjacent to some node in the subset.
    \\(c) \(E = \set{\langle G\rangle \mid \text{G is an undirected graph that contains a Euler circuit}}\).
    \\NOTE: An Euler circuit of the graph is a simple cycle that includes all edges.
\end{enumerate}
% -------------
\end{document}
