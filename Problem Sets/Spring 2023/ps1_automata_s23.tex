\documentclass[11pt]{article}
\usepackage{hyperref}
\usepackage{latexsym}
\usepackage{amsmath}
\usepackage{amssymb}
\usepackage{amsthm}
\usepackage{epsfig}
\usepackage{amsfonts}
\usepackage{braket}


\newcommand{\handout}[5]{
  \noindent
  \begin{center}
  \framebox{
    \vbox{
      \hbox to 5.78in { {\bf CSE 309 Theory of Automata} \hfill #2 }
      \vspace{4mm}
      \hbox to 5.78in { {\Large \hfill #5  \hfill} }
      \vspace{2mm}
      \hbox to 5.78in { {\em #3 \hfill #4} }
    }
  }
  \end{center}
  \vspace*{4mm}
}

\newcommand{\lecture}[4]{\handout{#1}{#2}{#3}{#4}{#1}}

\newtheorem{theorem}{Theorem}
\newtheorem{corollary}[theorem]{Corollary}
\newtheorem{lemma}[theorem]{Lemma}
\newtheorem{observation}[theorem]{Observation}
\newtheorem{proposition}[theorem]{Proposition}
\newtheorem{definition}[theorem]{Definition}
\newtheorem{claim}[theorem]{Claim}
\newtheorem{fact}[theorem]{Fact}
\newtheorem{assumption}[theorem]{Assumption}

\topmargin 0pt
\advance \topmargin by -\headheight
\advance \topmargin by -\headsep
\textheight 8.9in
\oddsidemargin 0pt
\evensidemargin \oddsidemargin
\marginparwidth 0.5in
\textwidth 6.5in

\parindent 0in
\parskip 1.5ex
%\renewcommand{\baselinestretch}{1.25}

\begin{document}

\lecture{Problem Set 1}{\textit{Assigned: Friday, 20 Jan}}{Spring 2023}{Due: \textit{11:59 pm Sunday, 5 Feb.}}

\centerline{{\Large To facilitate grading and timely feedback}}
\centerline{{\Large please note that all submissions are through \href{http://gradescope.com}{Gradescope}.}}
\centerline{{\Large Solve each problem on a new page.}}
\centerline{{\Large Plagiarized solutions, verbatim copies even with proper citation will receive zero credit.}}

\begin{enumerate}
\item Let $L = \{ w \in \{0,1,2\}^* \, | \, w \textrm{ represents an integer in ternary that is divisible by } 5\}$. Find an automaton for $L$.

\item Let $\Sigma = \{a,b\}$. For each $k \geq 1$, let $L_k$ be the language consisting of all strings that contain an $a$ exactly $k$ places from the right-hand end of the string. Thus, $L_k = \Sigma^* a \Sigma^{k-1}$. Describe an NFA with $k+1$ states that recognizes $L_k$ using only $k+1$ states. Prove that for each $k$, no DFA can recognize $L_k$ with fewer than $2^k$ states. Thus, NFAs can give an exponential improvement over DFAs in terms of memory requirement.

\item What is the smallest NFA you can design for $\{ a^n : n \ne 1003\}$. You should reason why your NFA is correct. 

\item We say a state $q \in Q$ is a \emph{dead end} if $\delta^*(q,w) \cap F = \emptyset$ for all $w \in \Sigma^*$. Argue that we can without loss of generality assume that if $q$ is a dead end, then it is also a \emph{trap state}. A trap state is one from which you can never escape. Argue that we can assume that a DFA has at most one trap state. Finally, show that a regular language $L$ is accepted by a DFA with no trap state if and only if \textsf{PREFIX}$(L) = \Sigma^*$.

\item If $w$ is a string then $VANISH(w)$ is all strings you can form by replacing some symbols in $w$ with the empty string. $VANISH(L)$ is defined in the obvious way. Prove or disprove that if $L$ is regular then $VANISH(L)$ is also regular.

\end{enumerate}
\end{document}
