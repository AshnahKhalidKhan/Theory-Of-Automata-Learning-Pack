\documentclass[11pt, article, oneside]{memoir}

% Some useful packages
\usepackage
{
  microtype,
  blindtext,
  multirow,
  makecell,
  graphicx,
  hyperref,
  nicematrix,
  mathtools,
  amsthm,
  amssymb,
  thmtools,
  bm,
  siunitx,
  caption
}

%% PDF Metadata
\hypersetup
{
  hidelinks,
  pdftitle    = Title,
  pdfauthor   = Ashnah Khalid Khan
}


% Page Layout for Printing A4
\settrimmedsize{297mm}{210mm}{*}
\setlength{\trimtop}{0pt}
\setlength{\trimedge}{\stockwidth}
\addtolength{\trimedge}{-\paperwidth}
\settypeblocksize{634pt}{448.13pt}{*}
\setulmargins{4cm}{*}{*}
\setlrmargins{*}{*}{1}
\setmarginnotes{17pt}{51pt}{\onelineskip}
\setheadfoot{\onelineskip}{2\onelineskip}
\setheaderspaces{*}{2\onelineskip}{*}
\checkandfixthelayout%


% Page Style
\makepagestyle{SD}
\nouppercaseheads
\makeoddhead{SD}{\textsc{Theory of Automata}}{}{\textsc{Summer 2023}}
\makeoddfoot{SD}{}{\thepage}{}
\makeheadrule{SD}{\textwidth}{\normalrulethickness}
% \makefootrule{SD}{\textwidth}{\normalrulethickness}
\pagestyle{SD}

% Paragraph Indent
\nonzeroparskip
\setlength{\parindent}{0pt}

% Command to make writing sets easier
\newcommand{\set}[1]{\{\, #1\, \}}

% Absolute Value Operator
\DeclarePairedDelimiter\abs{\lvert}{\rvert}

\begin{document}
% -------------
\begin{center}
\LARGE{\textsc{Problem Set 2}}
 
\large{\textsc{Ms. Asma Sanam Larik}}
\end{center}

\begin{enumerate}
    % 1.
    \item
        Let \(\Sigma = \set{a, b, c}\) and \(L = \set{ {a^i}{b^j}{c^k} \mid i, j, k \ge 0 \text{ and if } i = 1,  \text{ then } j = k}\). Show that \textit{L} is not regular using the pumping lemma.
        \\\textbf{Hint:} You must show both cases where when \(w = xyz\) that \(x = \lambda\) and where \(x\) is composed of some characters.

        \textbf{Solution:}
        \\Let \(w = xyz\) and \(w = ab^pc^p\).
        \\If \(x = \lambda, y = ab^m\) and \(z = b^{p - m}c^p\),
        \\then \(w_i = \lambda(a^m)^i(b^{p - m}c^p)\).
        \\Pump up to get \(w_2 = (ab^m)^2(b^{p - m}c^p)
        \\= ab^mab^mb^{p - m}c^p
        \\= ab^mab^pc^p\)
        \\which does not belong to the language so we have a contradiction.
        \\And if \(x = ab^n, y = b^m\) and \(z = b^{p - m - n}c^p\),
        \\then \(w_i = (ab^n)(b^m)^i(b^{p - m - n}c^p)\).
        \\Pump up to get \(w_2 = (ab^n)(b^m)^2(b^{p - m - n}c^p)
        \\= ab^nb^mb^mb^{p - m - n}c^p
        \\= ab^{p + m}c^p\)
        \\which does not belong to the language so we have a contradiction.

        \textbf{Grading Scheme:}
        \\+1 mark - Chose a string \(w\) that belongs to the language
        \\+1 mark - Decompose \(w\) into \(w = xyz\) without 'fixing' any constraints
        \\+1 mark - Pump up/down \(w_i = xy_iz\)
        \\+1 mark - Show valid contradiction for \(x = \lambda\)
        \\+1 mark - Show valid contradiction for \(x \neq \lambda\)
        
    % 2.
    \item 
        Let \(\Sigma = \set{0, 1, +, =}\) and \(ADD = \{ u+v=w \mid u, v, w \in \set{0, 1}^* \text{ and } u+v=w \text{ is a valid addition operation}\}\). Show that \textit{ADD} is not regular using the pumping lemma.
        
        \textbf{Solution:}
        \\\(w\) can be any string which is a valid addition, for example:
        \\Let \(w = xyz\) and \(w => {1^p+0^p=1^p}\).
        \\And let \(x = 1^j, y = 1^k\) where \(|xy| \leq p\), and \(z => {1^{p - j - k}+0^p=1^p}\).
        \\Then \(w_i => {(1^j)(1^k)^i(1^{p - j - k}+0^p=1^p)}\).
        \\Pump up to get \(w_2 => {(1^j)(1^k)^2(1^{p - j - k}+0^p=1^p)}
        \\=> {1^j1^k1^k1^{p - j - k}+0^p=1^p}
        \\=> {1^{p+k}+0^p=1^p}\)
        \\which is not a valid addition operation so it does not belong to the language so we have a contradiction.

        \textbf{Grading Scheme:}
        \\+1 mark - Chose a string \(w\) that belongs to the language
        \\+1 mark - Decompose \(w\) into \(w = xyz\) without 'fixing' any constraints
        \\+1 mark - Pump up/down \(w_i = xy_iz\)
        \\+2 mark - Show valid contradiction
        
    % 3.
    \item
        For each of the following grammars, determine if they are ambiguous or unambiguous. If they are ambiguous, prove that by giving two parse trees for one same string.
        \begin{enumerate}
            % a)
            \item
                \(S \rightarrow XY \mid W \)
                \\\(X \rightarrow aXb \mid \lambda\)
                \\\(Y \rightarrow cY \mid \lambda\)
                \\\(W \rightarrow aWc \mid Z\)
                \\\(Z \rightarrow bZ \mid \lambda\)
            % b)
            \item
                \(S \rightarrow XX \)
                \\\(X \rightarrow aXb\)
                \\\(Y \rightarrow cY \mid \lambda\)
            % c)
            \item
                \(S \rightarrow aXY \mid bYX \mid \lambda \)
                \\\(Z \rightarrow aZ \mid a\)
                \\\(X \rightarrow aXY \mid a \lambda\)
                \\\(Y \rightarrow bYZ \mid b \mid \lambda\) 
        \end{enumerate}

        \textbf{Solution:}
        \\Demonstrated in tutorial

        \textbf{Grading Scheme:}
        \\+0.5 marks - Attempt at making any regular expression for
        
    % 4.
    \item
        Construct a CFG for the following language over \(\Sigma = \set{\texttt{a}, \texttt{b}, \texttt{c}}\):
        \begin{enumerate}
            % a)
            \item
                \(A = \set{xby \mid x, y \in \{a\}^*}\) 
            % b)
            \item
                \(B = \set{ {a^i}{b^j}{c^k} \mid j = i + k}\)
            % c)
            \item
                \(C = \set{ {a^i}{b^j}{c^k} \mid j = i \text{ or } j = k}\)
        \end{enumerate}
                
        \textbf{Solution:}
        \\

        \textbf{Grading Scheme:}
        \\+0.5 marks - Attempt at making any regular expression for
        
    % 5.
    \item
        Construct push-down automata for the following languages:
        \begin{enumerate}
            % a)
            \item
                \(A = \set{w \mid w \text{ is a palindrome i.e. } w = w^R \text{ and } w, w^R \in \{0, 1\}^*}\) 
            % b)
            \item
                \(B = \set{ {0^i}{1^j}{2^k} \mid i = j + k}\)
        \end{enumerate}

        \textbf{Solution:}
        \\Demonstrated in tutorial

        \textbf{Grading Scheme:}
        \\+0.5 marks - Attempt at making any regular expression for
        
\end{enumerate}
% -------------
\end{document}
